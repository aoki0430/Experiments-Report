\documentclass{jsarticle}
\begin{document}
\title{振動の計測と制御}
\author{青木良太}
\maketitle


\section{圧電素子式の加速度センサの特性}
\subsection{目的}
機械システムの特性や性能を評価する方法の一つに、時間的に応答が速い"動特性"がある. \\*今回は, 振動現象のモデル化と数学的取り扱い方法, 振動の計測方法, 機械構造の振動挙動, FFTアナライザを用いた振動解析方法の内容を理解し, 加えてこれらの内容から動特性を理解することを目的とする. 
\subsection{実験装置、計測機器および解析装置}
画像
\subsection{実験の手順}
こーやった
\subsection{結果}
画像
\subsection{考察}
\subsubsection{入力信号[力]の周波数が変化すると、出力信号[電圧]にはどのような関係や特徴があったのか? }
\subsubsection{この圧電素子の与えられた力と発生する電圧の変換係数はいくらと判定できるか? }
ヒント: ただし、加速度センサ内の質量mは1.5(g)とし、加振器の振幅は、振動部に取り付けた鉛筆の動きを紙に転写して、大体の大きさ2hを求めてください。

\section{1自由度の強制振動系の挙動 }
\subsection{目的}
\subsection{実験装置、計測機器および解析装置}
画像
\subsection{実験の手順}
こーやった
\subsection{結果}
画像
\subsection{考察}
\subsubsection{このばね振動系の質量は何g であったか(装置に書いてある)?}
\subsubsection{得られた共振周波数からばね係数はいくらと推定できるか? }
\subsubsection{静的な荷重から求めたばね係数と共振周波数から推定したばね係数がどれくらい違いがあり、その理由はどのようなことが考えられるか? }\noindent
ヒント1: もちろん測定の誤差もありますが、ばね自体の質量を考慮した1自由度の振動系を考えることも必要ですね。\\*
ヒント2: 静的なばね定数の測定実験による結果をグラフにすると特性が非線形であることがわかります。実際の振動現象が原点付近の0.25mm程度の範囲で起こっていることを考えると静的なばね定数をいくらとすべきであるか?
\subsubsection{<実験データA>は各周波数での加振器の振動、<実験データB>は各周波数での機械振動系の振動であり、これらを比較して、どのような事が言えるか?}

\section{1自由度振動系のインパルス応答のFFT 解析}
\subsection{目的}
\subsection{実験装置、計測機器および解析装置}
画像
\subsection{実験の手順}
こーやった
\subsection{結果}
\begin{itemize}
	\item インパクトハンマーの信号と振動系の加速度センサー信号の時間軸波形(データ011.bmp)
	\item インパクトハンマーから得られた信号Gxのパワースペクトルと1自由度振動系に取りつけた加速度センサからの信号Gyのパワースペクトル(データ 013.bmp)
	\item 伝達関数H = Gy/Gxの周波数応答スペクトル(周波数領域における入出力比)のグラフ(データ 015.bmp)
	\item (これらのデータをUSBメモリに記録し、WORDなどで編集し、データの内容を説明し、PDF1ページで提出すること)
\end{itemize}
\subsection{考察}
\subsubsection{入力信号[力]の周波数が変化すると、出力信号[電圧]にはどのような関係や特徴があったのか? }
\subsubsection{この圧電素子の与えられた力と発生する電圧の変換係数はいくらと判定できるか? }
ヒント: ただし、加速度センサ内の質量mは1.5(g)とし、加振器の振幅は、振動部に取り付けた鉛筆の動きを紙に転写して、大体の大きさ2hを求めてください。


\section{おわりに}

これは一段組の例ですが,二段組に変更することもできます。

解説文を読んで,このソースをいろいろと変更してみましょう。

\end{document}



