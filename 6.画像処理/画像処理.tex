\documentclass[a4paper,11pt,uplatex]{jsarticle}
\usepackage{amsmath,amssymb}
\usepackage{bm}
\usepackage[dvipdfmx]{graphicx}
\usepackage{here}

%テキストの表示領域の調節
\setlength{\textwidth}{\paperwidth}
\addtolength{\textwidth}{-40truemm}
\setlength{\textheight}{\paperheight}
\addtolength{\textheight}{-45truemm}

%余白の調節
\setlength{\topmargin}{-10.4truemm}
\setlength{\evensidemargin}{-5.4truemm}
\setlength{\oddsidemargin}{-5.4truemm}
\setlength{\headheight}{17pt}
\setlength{\headsep}{10mm}
\addtolength{\headsep}{-17pt}
\setlength{\footskip}{5mm}

\begin{document}
\section{目的}
画像処理は工学の様々な分野で用いられており, 機械系技術者にとっても避けては通れない技術である.
最近は画像処理のパッケージ・ソフトウェアも一般的となり, 様々な処理が手軽に行えるようになっている.
しかし, 自らが新たなシステムを開発する場合はもちろん, 既成のソフトウェアを利用するときにおいても,
処理内容の意味を正しく理解した上で使うことが大切である. ここでは, いくつかの簡単なデジタル画像処理をコンピュータ上で行い,
その基本原理と特性を理解することを目的とする.
\section{理論}
\subsubsection{画像処理}
画像処理(image processing)とは画像で表現された情報の処理の総称であり, 図hogeに示すように様々な内容が含まれる.
ただし, 狭義には, 2次元画像を処理して新たな2次元画像を生成したり, 画像から何らかの情報取り出したりすること意味し,
歪の修正, ぼけやノイズの除去, 特定成分の強調などの画像変換やパターンを挿す場合も多い. 画像処理にコンピュータを利用すると便利なことが多いが,
決して必須ではない. 例えば, 写真の現像は化学的方法によって, 偏光フィルタによる成分抽出は光学的方法によって行われる.
また, 種々の処理を専用の電子回路によって行うことも可能であるし, 生物が非常に高度な画像処理を瞬時に行なっていることは言うまでもないだろう.
\subsubsection{画像のディジタル化}
現在一般に使われているノイマン型のコンピュータは, 自然界に存在する連続量をそのままの形で取り扱うことができない.
そこで, コンピュータで画像処理を行う場合, 画像を標本化(sampling)と量子化(quantization)と言う二つのプロセスによって離散化し,
ディジタル(digital)量として表現する. 標本化とは時間的あるいは空間的に連続して変化する信号から,
その軸上のいくつかの離散点における値を取り出す操作のことで, 画像情報においては画面を正方形や六角形などの画素(pixel)に区切ることで実現される.
また, 量子化は各画素内の明るさを代表する値を階調値(gray level)と呼ばれる有限個のレベルのどれかに割り当てることで実現される.
これらの操作によって表現された画像をディジタル画像(digital image)という. すなわち, ディジタル画像とは図に示すように,
明るさに対応した離散量をもつ有限個の画素2次元に集まって構成された画像のことである.
ディジタル画像は階調値の取る個数(階調数)によって表1のように分類することができる.
\subsection{画像演算}
一つのディジタル画像に何らかの処理を施して別の画像を得ることを, 画像演算という.
今, ディジタル画像全体の集合を$D$,$D$の部分集合を$D_1,D_2,D_3$と書けば, 画像 $F \in D_1, G \in D_2, H \in D_3$
に対して,
\begin{align}
  G = O(F) \\
  H = \Phi(F,G)
\end{align}
という演算を考えることができる. 式(1)における写像$O: D_1 \to D_2$を単項演算, 式(2)における写像$\Phi: D_1 \times D_2 \to D_3$
を二項演算という. 画像二項演算には実数演算と同様に, 論理演算, 四則演算, Max/Min演算が定義される.
\subsection{フィルタリング}
画像のフィルタリング(filterling)とは, 任意の画素の新しい階調値を注目画素の近傍(neighborhood)内に
ある画素の階調値から決定する局所演算(local operation)を画像上のすべての画素に対して行う処理のことである.
このとき, 各画素に対する演算はその位置によらず独立して行われ, 近傍の範囲は画像全体に比べて十分小さい. このような近傍は範囲をマスクと呼び,
フィルタリングにおいては$3\times3, 5\times5, 7\times7$などのマスクが適用される. フィルタリングは画像認識などの前処理として用いられることが多い.
\subsubsection{平滑化フィルタ}
注目画素の近傍内にある画素の階調値の平均値を求め, それを注目がその新しい階調値にすることを考える. 新しい画像を$G = g(i,j)$とすると, 求める式は次のように表される.
\begin{align}
  g(i,j) = \frac{1}{9} \sum_{m=-1}^{1} \sum_{n=-1}^{1} f(i+m, j+n)
\end{align}
このような演算を行うフィルタは, 画像中の階調値の変化を滑らかにしてノイズを除去する機能を持つことから平滑化フィルタと呼ばれる. フィルタは演算に用いる
係数をマトリクス状に並べて表示することができる. 例えば, 式(3)のフィルタは図hogeに示されるような$3\times3$マスクに
よって表される. なお, 図hogeのように注目画素の重みを大きくした平滑化フィルタも存在する. 図hogeのフィルタを数式で表すと,
式(4)のようになる.
\begin{align}
  g(i,j) = \frac{1}{10} { \sum_{m=-1}^{1} \sum_{n=-1}^{1} f(i+m, j+n)+f(i,j) }
\end{align}

\subsubsection{差分型フィルタ}
差分型フィルタ(difficult filter)は, 注目画素の近傍の階調値の微分(差分)をとるフィルタで,
画像中の線や縁など階調変化の急な部分を抽出することができる. 例として, 水平方向の微分$\partial f / \partial j$は$3\times3$
マスクを用いれば,
\begin{align}
  g(i,j)=-f(i-1,j-1)+f(i+1,j-1)-f(i-1,j)+f(i+1,j)-f(i-1,j+1)+(i+1,j+1)
\end{align}
のように求められる(図hoge). 同様の手順により, 垂直方向の微分$\partial f / \partial j$を表す$3\times3$マスクのフィルタを得ることができる.
\subsubsection{ラプラシアンフィルタ}
二次の空間微分であるラプラシアン(Laplacian)
\begin{align}
  \Delta = \nabla^{2} = \frac{\partial^{2}}{\partial j^{2}}
\end{align}
は, 画像中の輪郭線を強調する機能を持っている. これをフィルタとして用いたものがラプラシアンフィルタであり,
$3\times3$マスクを用いた4近傍に対するラプラシアンフィルタは図のように表される.
\subsection{点演算}
画像中のある領域の画素の階調値を問題にする時, 注目画素における階調値をそのまま用いるのではなく,
何らかの変換を行った値を用いたほうが便利なことがある. 点演算(point operation)はこのような場合に行われる処理で,
新しい画像上の画素の階調値を元の画像の同じ位置における階調値から決定するものである.
\subsubsection{二値化}
設定したしきい値(threshold level)よりも小さい階調値を持つ画素を0(黒), 大きい階調値を持つ画素を
1(白)にすることで, 階調画像を二値画像に変換する処理のことを二値化(binarization)という.
しきい値には, 濃度ヒストグラム(横軸に階調値, 縦軸に画素数をとったグラフ)の谷部分や, 二分割される画素数が適当な割合になる位置などが選ばれるが,
結果に大きな影響を与えるので注意する.
\subsubsection{階調変換}
適当な関数を定義することで, 画像中の階調値の分布範囲や分布関数を目的のものに変える処理が階調変換(gray level conversion)である.
例えば, 階調値$L$がある狭い範囲$[L_1,L_2]$にしか分布していない画像は, コントラストが不明瞭になっていることが多い.
このような場合, 原画像に次式で定義される線形変換を施して階調変化の範囲を$[L_{min},L_{max}]$まで広げれば, 画像のコントラストを明瞭にすることができる.
\begin{align}
  L^{*} = \frac{L_{max}-L_{min}}{L_2-L_1}(L-L_1)+L_{min}
\end{align}
ここで, $L^{*}$は変換後の階調値である.

\section{実験方法}


\section{実験結果及び考察}

\section{結論}

\section{参考文献}

\end{document}
